\begin{Shaded}
\begin{Highlighting}[]
\FunctionTok{Campus}\KeywordTok{:}\AttributeTok{ Ciudad Universitaria}
\FunctionTok{Facultad}\KeywordTok{:}\AttributeTok{ Ingeniería}
\FunctionTok{Materia }\KeywordTok{:}\AttributeTok{ Inteligencia Artificial}
\FunctionTok{Semestre}\KeywordTok{:}\AttributeTok{ 2022{-}2}
\FunctionTok{Equipo}\KeywordTok{:}\AttributeTok{ }\DecValTok{1}
\FunctionTok{Clave}\KeywordTok{:}\AttributeTok{ }\DecValTok{0406}
\FunctionTok{Participantes}\KeywordTok{:}\AttributeTok{ }
\AttributeTok{    }\KeywordTok{{-}}\AttributeTok{ Barrera Peña Víctor Miguel}
\AttributeTok{    }\KeywordTok{{-}}\AttributeTok{ Espino De Horta Joaquín Gustavo}
\AttributeTok{    }
\FunctionTok{Profesor}\KeywordTok{:}\AttributeTok{ Dr. Ismael Everardo Barcenas Patiño}
\FunctionTok{Título }\KeywordTok{:}\AttributeTok{ Proyecto 2}
\FunctionTok{Subtítulo }\KeywordTok{:}\AttributeTok{ Fórmula de lógica proposicional valida o no}
\end{Highlighting}
\end{Shaded}

\includegraphics{img/documentacion/1.png}

\break

\tableofcontents

\break

\hypertarget{definiciuxf3n-del-problema}{%
\section{Definición del problema}\label{definiciuxf3n-del-problema}}

Escribir un programa lógico que determine si una fórmula de lógica
proposicional es válida o no.

• Algoritmo basado en el cálculo de secuentes. • Entrada: una expresión
de lógica proposicional. • Salida: una prueba (en código latex), si
existe, de lo contrario, no válida.

\hypertarget{capuxedtulo-0-estructura-del-repositorio}{%
\section{Capítulo 0 Estructura del
repositorio}\label{capuxedtulo-0-estructura-del-repositorio}}

El repositorio documenta todo el proceso hecho para realizar el programa
, incluyendo los pdf´s

\hypertarget{bibliotecas-usadas}{%
\subsection{Bibliotecas usadas}\label{bibliotecas-usadas}}

\begin{Shaded}
\begin{Highlighting}[]
\FunctionTok{\textbackslash{}usespackage}\NormalTok{\{prof\}}
\FunctionTok{\textbackslash{}usespackage}\NormalTok{\{bussproofs\}}
\end{Highlighting}
\end{Shaded}

Para poder ejecutar el programa es necesario tener instalado un
interprete de \texttt{prolog} para poder ejecutar el programa.

Para dar la apariencia a las etiquetas de los \texttt{prooftree} pues se
uso

\begin{Shaded}
\begin{Highlighting}[]
\FunctionTok{\textbackslash{}mathcal}\NormalTok{\{I\} }\FunctionTok{\textbackslash{}mathcal}\NormalTok{\{D\}}
\end{Highlighting}
\end{Shaded}

\hypertarget{capuxedtulo-1-introducciuxf3n}{%
\section{Capítulo 1 Introducción}\label{capuxedtulo-1-introducciuxf3n}}

``Aquel modo de razonamiento que examina el estado de las cosas
afirmadas en las premisas, forma un diagrama de este estado de cosas,
percibe en las partes de ese diagrama relaciones no mencionadas
expl´ıcitamente en las premisas, se satisface a s´ı mismo mediante
experimentos mentales sobre el diagrama, de que estas relaciones siempre
subsistir´an, y concluye su verdad necesaria o probable'' Charles
Sanders Peirce.

Razonar partiendo de premisas mediante reglas para llegar a conclusiones
es la tarea que tomamos desde la matemática para poder demostrar, este
es uno de los principales objetivos de la lógica. Formalmente se
demuestra la validez de un razonamiento mediante el método de deducción.

El método de la deducción se hace usando las leyes de lógica de primer
orden o mejor conocidas como \textbf{deducción natural}. Para logar
obtener un resultado se hace uso de su principal herramienta que es el
\textbf{cálculo de secuentes}.

``El cálculo de secuentes presenta numerosas analogía con la deducción
natural y, aunque se adapta bien al caso intuicionista, también resuelve
problemas de validez y consecuencias lógicas para la lógica
proposicional clásica.''

\hypertarget{capuxedtulo-2-desarrollo}{%
\section{Capítulo 2 Desarrollo}\label{capuxedtulo-2-desarrollo}}

\hypertarget{idea-de-desarrollo-del-programa}{%
\subsection{Idea de desarrollo del
programa}\label{idea-de-desarrollo-del-programa}}

\hypertarget{casos-de-prueba}{%
\subsection{Casos de prueba}\label{casos-de-prueba}}

\hypertarget{triviales-1-caso}{%
\subsubsection{Triviales (1 caso)}\label{triviales-1-caso}}

\begin{prooftree}
\RightLabel{ $\mathcal{I} \mathcal{D}$}
 \AxiomC{}
 \UnaryInfC{$A \vdash A $}
\end{prooftree}

\hypertarget{fuxe1ciles-3-casos}{%
\subsubsection{Fáciles (3 casos)}\label{fuxe1ciles-3-casos}}

\hypertarget{caso-1}{%
\paragraph{Caso 1}\label{caso-1}}

\begin{prooftree}
\RightLabel{ $\mathcal{I} \mathcal{D}$}
\AxiomC{}
\UnaryInfC{$\vdash \neg A,A \:$}
\RightLabel{ $\mathcal{D}\neg$}
\UnaryInfC{$\vdash \neg A,A \:$}
\RightLabel{ $\mathcal{I}\neg$}
\UnaryInfC{ $\neg \neg A \vdash A$ }
\end{prooftree}

\hypertarget{caso-2}{%
\paragraph{Caso 2}\label{caso-2}}

\begin{prooftree}
\AxiomC{$$}
\RightLabel{ $\mathcal{I} \mathcal{D}$}
\UnaryInfC{$A \vdash A$}
\AxiomC{$$}
\RightLabel{ $\mathcal{I} \mathcal{D}$}
\UnaryInfC{$B \vdash B$}
\RightLabel{ R ∧}
\BinaryInfC{$A, B \vdash A ∧B$}
\end{prooftree}

\hypertarget{caso-3}{%
\paragraph{Caso 3}\label{caso-3}}

\begin{prooftree}
\AxiomC{$$}
\RightLabel{ $\mathcal{I} \mathcal{D}$}
\UnaryInfC{$A \vdash A$}
\RightLabel{ $\mathcal{R} \neg$}
\UnaryInfC{$ \vdash \neg A,A$}
\RightLabel{ $\mathcal{RX} $}
\UnaryInfC{$ \vdash A ,\neg A$}
\RightLabel{ $\mathcal{ANT}$}
\UnaryInfC{$ \vdash A ∨ \neg A$}
\end{prooftree}

\hypertarget{media-3-casos}{%
\subsubsection{Media (3 casos)}\label{media-3-casos}}

\hypertarget{caso-1-1}{%
\paragraph{Caso 1}\label{caso-1-1}}

\begin{prooftree}


\AxiomC{$B \vdash A,A$}
\AxiomC{$B,B, \vdash A$}

\RightLabel{ $\mathcal{I} \rightarrow $}
\BinaryInfC{$B,A \rightarrow B \vdash A$}

\RightLabel{ $\mathcal{D} \rightarrow  $}
\UnaryInfC{$A \rightarrow  B \vdash B \rightarrow A $}

\RightLabel{ $I \rightarrow$}
\AxiomC{$A \vdash B, B$}
\RightLabel{ $ I\rightarrow$}
\AxiomC{$ A,A \vdash B$}

\RightLabel{ $\mathcal{I} \rightarrow$}
\BinaryInfC{$A,B \rightarrow A \vdash B$}

\RightLabel{ $\mathcal{D} \rightarrow$}
\UnaryInfC{$B \rightarrow A \vdash A \rightarrow B $}

\RightLabel{ $D \leftrightarrow$ }
\BinaryInfC{$ \vdash (A ∨ B) \rightarrow (A  \leftrightarrow B)$}
\end{prooftree}

\hypertarget{caso-2-1}{%
\paragraph{Caso 2}\label{caso-2-1}}

\begin{prooftree}

\AxiomC{$$}

\RightLabel{ $\mathcal{I} \mathcal{D}$}
\UnaryInfC{$A \vdash A$}
\AxiomC{$A \vdash B$}

\RightLabel{ $\mathcal{D} ∧ $}
\BinaryInfC{$A \vdash (A ∧ B)$}

\AxiomC{$$}

\RightLabel{ $\mathcal{I} \mathcal{D}$}
\UnaryInfC{$A \vdash A$}

\RightLabel{ $\mathcal{I}- Debilitamiento$}
\UnaryInfC{$A,B \vdash A$}

\RightLabel{ $\mathcal{I} ∧$}
\UnaryInfC{$(A ∧ B) \vdash A$ }

\RightLabel{ $\mathcal{D} \leftrightarrow$ }
\BinaryInfC{$\vdash A \leftrightarrow (A ∧ B )$}
\end{prooftree}

\hypertarget{caso-3-1}{%
\paragraph{Caso 3}\label{caso-3-1}}

\begin{prooftree}

\AxiomC{$\vdash A,B$}

\RightLabel{ $ \mathcal{I} \neg$}
\UnaryInfC{$\neg A \vdash B$ }

\RightLabel{ $ \mathcal{D} ∨$}
\UnaryInfC{$ \vdash (A ∨ \neg B)$ }

\AxiomC{$A,B\vdash$}

\RightLabel{ $\mathcal{D} \neg$}
\UnaryInfC{$A \vdash  \neg B$}

\RightLabel{ $ \mathcal{D} ∨$}
\UnaryInfC{$ \vdash (A ∨ \neg B)$ }

\RightLabel{ $\mathcal{D} \hspace{0.3cm} Conjugacion $ }
\BinaryInfC{$\vdash(\neg A ∨ B) ∧( A ∨ \neg B) $}
\end{prooftree}

\hypertarget{difuxedciles-3-casos}{%
\subsubsection{Difíciles ( 3 casos )}\label{difuxedciles-3-casos}}

\hypertarget{caso-1-2}{%
\paragraph{Caso 1}\label{caso-1-2}}

(revisar)

\begin{prooftree}

\AxiomC{$\vdash A,B$}

\RightLabel{ $ \mathcal{I} \neg$}
\UnaryInfC{$\neg A \vdash B$ }

\RightLabel{ $ \mathcal{D} ∨$}
\UnaryInfC{$ \vdash (A ∨ \neg B)$ }

\AxiomC{$A \vdash A$}

\RightLabel{ $\mathcal{D} \neg$}
\UnaryInfC{$A, A \rightarrow B \vdash B$}

\RightLabel{ $ \mathcal{D} ∨$}
\UnaryInfC{$ A \rightarrow \vdash A \rightarrow B$ }

\RightLabel{ $\mathcal{D} \rightarrow$ }
\BinaryInfC{$\vdash( A ∨ B) \rightarrow( A \leftrightarrow B) $}
\end{prooftree}

\hypertarget{caso-2-2}{%
\paragraph{Caso 2}\label{caso-2-2}}

Caso 3

\hypertarget{sin-soluciuxf3n-1-caso}{%
\subsubsection{Sin solución (1 caso)}\label{sin-soluciuxf3n-1-caso}}

\begin{prooftree}
\RightLabel{ $\mathcal{I} \mathcal{D}$}
 \AxiomC{}
 \UnaryInfC{$A \vdash B $}
\end{prooftree}

\hypertarget{caso-1-3}{%
\paragraph{Caso 1}\label{caso-1-3}}

\hypertarget{cuxf3digo}{%
\subsubsection{Código}\label{cuxf3digo}}

\begin{Shaded}
\begin{Highlighting}[]
\FunctionTok{//}\DataTypeTok{Definir}\NormalTok{ operaciones}\FunctionTok{//}

\KeywordTok{:{-}}\FunctionTok{op}\NormalTok{(}\DecValTok{1}\KeywordTok{,}\NormalTok{fx}\KeywordTok{,}\NormalTok{neg)}\KeywordTok{.}
\KeywordTok{:{-}}\FunctionTok{op}\NormalTok{(}\DecValTok{2}\KeywordTok{,}\NormalTok{xfy}\KeywordTok{,}\NormalTok{or)}\KeywordTok{.}
\KeywordTok{:{-}}\FunctionTok{op}\NormalTok{(}\DecValTok{2}\KeywordTok{,}\NormalTok{xfy}\KeywordTok{,}\NormalTok{and)}\KeywordTok{.}
\KeywordTok{:{-}}\FunctionTok{op}\NormalTok{(}\DecValTok{2}\KeywordTok{,}\NormalTok{xfy}\KeywordTok{,}\NormalTok{imp)}\KeywordTok{.}
\KeywordTok{:{-}}\FunctionTok{op}\NormalTok{(}\DecValTok{2}\KeywordTok{,}\NormalTok{xfy}\KeywordTok{,}\NormalTok{dimp)}\KeywordTok{.}

\FunctionTok{//}\DataTypeTok{Win} \DataTypeTok{Condition}
\NormalTok{sq(}\DataTypeTok{G}\KeywordTok{,}\DataTypeTok{D}\NormalTok{)}\KeywordTok{:{-}}
\NormalTok{sq([}\DataTypeTok{F}\NormalTok{]}\KeywordTok{,}\NormalTok{[}\DataTypeTok{F}\NormalTok{])}\KeywordTok{.}

\FunctionTok{//}\DataTypeTok{Debilitamiento}
\NormalTok{sq([}\DataTypeTok{F}\FunctionTok{|}\DataTypeTok{G}\NormalTok{]}\KeywordTok{,}\DataTypeTok{D}\NormalTok{)}\KeywordTok{:{-}} \DataTypeTok{atom}\NormalTok{(}\DataTypeTok{F}\NormalTok{)}\KeywordTok{,}
\NormalTok{sq(}\DataTypeTok{G}\KeywordTok{,}\DataTypeTok{D}\NormalTok{)}\KeywordTok{.}

\NormalTok{sq(}\DataTypeTok{G}\KeywordTok{,}\NormalTok{[}\DataTypeTok{F}\FunctionTok{|}\DataTypeTok{D}\NormalTok{])}\KeywordTok{:{-}} \DataTypeTok{atom}\NormalTok{(}\DataTypeTok{F}\NormalTok{)}\KeywordTok{,}
\NormalTok{sq(}\DataTypeTok{G}\KeywordTok{,}\DataTypeTok{D}\NormalTok{)}\KeywordTok{.}

\FunctionTok{//}\DataTypeTok{Negacion}
\NormalTok{sq([neg }\DataTypeTok{F}\FunctionTok{|}\DataTypeTok{G}\NormalTok{]}\KeywordTok{,}\DataTypeTok{D}\NormalTok{)}\KeywordTok{:{-}}
\NormalTok{sq(}\DataTypeTok{G}\KeywordTok{,}\NormalTok{[}\DataTypeTok{F}\FunctionTok{|}\DataTypeTok{D}\NormalTok{])}\KeywordTok{,}
\FunctionTok{//write}\NormalTok{(}\DataTypeTok{Archivo}\KeywordTok{,}\OtherTok{"}\CharTok{\textbackslash{}n}\ErrorTok{eg}\OtherTok{"}\FunctionTok{+}\DataTypeTok{G}\NormalTok{)}\KeywordTok{.}

\NormalTok{sq(}\DataTypeTok{G}\KeywordTok{,}\NormalTok{[neg }\DataTypeTok{F}\FunctionTok{|}\DataTypeTok{D}\NormalTok{])}\KeywordTok{:{-}}
\NormalTok{sq([}\DataTypeTok{F}\FunctionTok{|}\DataTypeTok{G}\NormalTok{]}\KeywordTok{,}\DataTypeTok{D}\NormalTok{)}\KeywordTok{.}

\FunctionTok{//}\DataTypeTok{Disyuncion}
\NormalTok{sq([}\DataTypeTok{F}\NormalTok{ or }\DataTypeTok{R}\FunctionTok{|}\DataTypeTok{G}\NormalTok{]}\KeywordTok{,}\DataTypeTok{D}\NormalTok{)}\KeywordTok{:{-}}
\NormalTok{sq([}\DataTypeTok{F}\FunctionTok{|}\DataTypeTok{G}\NormalTok{]}\KeywordTok{,}\DataTypeTok{D}\NormalTok{)}\KeywordTok{,}
\NormalTok{sq([}\DataTypeTok{R}\FunctionTok{|}\DataTypeTok{G}\NormalTok{]}\KeywordTok{,}\DataTypeTok{D}\NormalTok{)}\KeywordTok{.}

\NormalTok{sq(}\DataTypeTok{G}\KeywordTok{,}\NormalTok{[}\DataTypeTok{F}\NormalTok{ or }\DataTypeTok{R}\FunctionTok{|}\DataTypeTok{D}\NormalTok{])}\KeywordTok{:{-}}
\NormalTok{append([}\DataTypeTok{F}\NormalTok{,}\DataTypeTok{R}\NormalTok{]}\KeywordTok{,}\DataTypeTok{D}\KeywordTok{,}\DataTypeTok{U}\NormalTok{)}\KeywordTok{,}
\NormalTok{sq(}\DataTypeTok{G}\KeywordTok{,}\DataTypeTok{U}\NormalTok{)}\KeywordTok{.}

\FunctionTok{//}\DataTypeTok{Conjuncion}
\NormalTok{sq([}\DataTypeTok{F}\NormalTok{ and }\DataTypeTok{R}\FunctionTok{|}\DataTypeTok{G}\NormalTok{]}\KeywordTok{,}\DataTypeTok{D}\NormalTok{)}\KeywordTok{:{-}}
\NormalTok{append([}\DataTypeTok{F}\NormalTok{,}\DataTypeTok{R}\NormalTok{]}\KeywordTok{,}\DataTypeTok{G}\KeywordTok{,}\DataTypeTok{U}\NormalTok{)}\KeywordTok{,}
\NormalTok{sq(}\DataTypeTok{U}\KeywordTok{,}\DataTypeTok{D}\NormalTok{)}\KeywordTok{.}

\NormalTok{sq(}\DataTypeTok{G}\KeywordTok{,}\NormalTok{[}\DataTypeTok{F}\NormalTok{ and }\DataTypeTok{R}\FunctionTok{|}\DataTypeTok{D}\NormalTok{])}\KeywordTok{:{-}}
\NormalTok{sq(}\DataTypeTok{G}\KeywordTok{,}\NormalTok{[}\DataTypeTok{F}\FunctionTok{|}\DataTypeTok{D}\NormalTok{])}\KeywordTok{,}
\NormalTok{sq(}\DataTypeTok{G}\KeywordTok{,}\NormalTok{[}\DataTypeTok{R}\FunctionTok{|}\DataTypeTok{D}\NormalTok{])}\KeywordTok{.}

\FunctionTok{//}\DataTypeTok{Implicacion}
\NormalTok{sq([}\DataTypeTok{F}\NormalTok{ imp }\DataTypeTok{R}\FunctionTok{|}\DataTypeTok{G}\NormalTok{]}\KeywordTok{,}\DataTypeTok{D}\NormalTok{)}\KeywordTok{:{-}}
\NormalTok{sq(}\DataTypeTok{G}\KeywordTok{,}\NormalTok{[}\DataTypeTok{F}\FunctionTok{|}\DataTypeTok{D}\NormalTok{])}\KeywordTok{,}
\NormalTok{sq([}\DataTypeTok{R}\FunctionTok{|}\DataTypeTok{G}\NormalTok{]}\KeywordTok{,}\DataTypeTok{D}\NormalTok{)}\KeywordTok{.}

\NormalTok{sq(}\DataTypeTok{G}\KeywordTok{,}\NormalTok{[}\DataTypeTok{F}\NormalTok{ imp }\DataTypeTok{R}\FunctionTok{|}\DataTypeTok{D}\NormalTok{])}\KeywordTok{:{-}}
\NormalTok{sq([}\DataTypeTok{F}\FunctionTok{|}\DataTypeTok{G}\NormalTok{]}\KeywordTok{,}\NormalTok{[}\DataTypeTok{R}\FunctionTok{|}\DataTypeTok{D}\NormalTok{])}\KeywordTok{.}

\FunctionTok{//}\DataTypeTok{Doble} \DataTypeTok{Implicacion}
\NormalTok{sq([}\DataTypeTok{F}\NormalTok{ dimp }\DataTypeTok{R}\FunctionTok{|}\DataTypeTok{G}\NormalTok{]}\KeywordTok{,}\DataTypeTok{D}\NormalTok{)}\KeywordTok{:{-}}
\NormalTok{append([}\DataTypeTok{F}\NormalTok{,}\DataTypeTok{R}\NormalTok{]}\KeywordTok{,}\DataTypeTok{D}\KeywordTok{,}\DataTypeTok{U}\NormalTok{)}\KeywordTok{,}
\NormalTok{sq(}\DataTypeTok{G}\KeywordTok{,}\DataTypeTok{U}\NormalTok{)}\KeywordTok{,}
\NormalTok{append([}\DataTypeTok{F}\NormalTok{,}\DataTypeTok{R}\NormalTok{]}\KeywordTok{,}\DataTypeTok{G}\KeywordTok{,}\DataTypeTok{U}\NormalTok{)}\KeywordTok{,}
\NormalTok{sq(}\DataTypeTok{U}\KeywordTok{,}\DataTypeTok{D}\NormalTok{)}\KeywordTok{.}

\NormalTok{sq(}\DataTypeTok{G}\KeywordTok{,}\NormalTok{[}\DataTypeTok{F}\NormalTok{ dimp }\DataTypeTok{R}\FunctionTok{|}\DataTypeTok{D}\NormalTok{])}\KeywordTok{:{-}}
\NormalTok{sq([}\DataTypeTok{F}\FunctionTok{|}\DataTypeTok{G}\NormalTok{]}\KeywordTok{,}\NormalTok{[}\DataTypeTok{R}\FunctionTok{|}\DataTypeTok{D}\NormalTok{])}\KeywordTok{,}
\NormalTok{sq([}\DataTypeTok{R}\FunctionTok{|}\DataTypeTok{G}\NormalTok{]}\KeywordTok{,}\NormalTok{[}\DataTypeTok{F}\FunctionTok{|}\DataTypeTok{D}\NormalTok{])}\KeywordTok{.}
\end{Highlighting}
\end{Shaded}

\hypertarget{explicaciuxf3n-cuxf3digo}{%
\subsection{Explicación código}\label{explicaciuxf3n-cuxf3digo}}

\hypertarget{capuxedtulo-3-conclusiuxf3n}{%
\section{Capítulo 3 Conclusión}\label{capuxedtulo-3-conclusiuxf3n}}

\hypertarget{barrera-peuxf1a-vuxedctor-miguel}{%
\subsection{Barrera Peña Víctor
Miguel}\label{barrera-peuxf1a-vuxedctor-miguel}}

Crear un programa en un nuevo paradigma de programación es un reto y más
si aquel lenguaje haz hecho pocos ejercicios, además de entrar en
conflicto con latex para crear las ecuaciones constituye un reto de buen
nivel. Por anteriores motivos puedo decir que si bien hasta el momento
no se ha logrado obtener todo a la perfección nos hemos acercado al
objetivo de crear un programa que pueda crear los secuentes usando
programación lógica.

En lo que respecta, posiblemente con una semana más y un poco de
práctica para encontrar el error de prolog es suficiente para poder
concluir el proyecto con exito.

\hypertarget{espino-de-horta-joaquuxedn-gustavo}{%
\subsection{Espino de Horta Joaquín
Gustavo}\label{espino-de-horta-joaquuxedn-gustavo}}

El mayor desafío de este proyecto sin duda era el conocer y manipular
correctamente el lenguaje argumentativo de Prolog un programa cuya única
función es la de comprobar hechos y operaciones lógicas. Si bien el
programa a realizar es en cierto punto sencillo, no se podía realizar de
la manera correcta por muchos factores de falta de información, donde no
se encuentra mucho soporte o documentación adecuada así como ciertas
funciones que nos acortaban trabajo.

Apuesto que si nos hubiesen dejado este mismo trabajo, con la gran
libertad de hacerlo con un lenguaje de programación de libre elección,
se habría llegado a un producto satisfactorio en menos de una semana
como si ocurrió con el proyecto anterior. En cuanto al objetivo, cumple
en el ámbito teórico así como su propuesta de ser posible a realizar en
las propiedades de Prolog.

Esperamos en los siguientes proyectos poder afrontarlos con la libertad
necesaria de herramientas o con un mayor conocimiento de las condiciones
a las que estaremos sujetos. Pues, lo que puedo argumentar, sería
realmente fácil este proyecto como lo sugiere el sentido común, aún con
la generación de archivo Latex si se permitiera usar cualquier lenguaje.

\hypertarget{anexos-para-investigaciuxf3n}{%
\section{Anexos para investigación}\label{anexos-para-investigaciuxf3n}}

\hypertarget{charles-sanders-peirce}{%
\subsection{Charles Sanders Peirce}\label{charles-sanders-peirce}}

\begin{itemize}
\tightlist
\item
  La ciencia de la
  semiótica,http://mastor.cl/blog/wp-content/uploads/2015/08/PEIRCE-CH.-S.-La-Ciencia-de-La-Semi\%C3\%B3tica.pdf
\end{itemize}

\hypertarget{fuentes}{%
\section{Fuentes}\label{fuentes}}

\begin{itemize}
\tightlist
\item
  FUENTES GUZMAN, D. A. N. I. E. L. C. A. M. I. L. O. (2014). CÁLCULO DE
  SECUENTES Y GRAFICOS EXISTENCIALES ALFA: DOS ESTRUCTURAS EQUIVALENTES
  PARA LA LOGICA PROPOSICIONAL. UNIVERSIDAD DEL TOLIMA. Recuperado 7 de
  abril de 2022, de
  http://repository.ut.edu.co/bitstream/001/1172/1/RIUT-ABA-spa-2014-C\%C3\%A1lculo\%20de\%20secuentes\%20y\%20gr\%C3\%A1ficos\%20existenciales\%20Alfa.\%20\%20Dos\%20estructuras\%20equivalentes\%20para\%20la\%20l\%C3\%B3gica\%20proposicional.pdf
\end{itemize}
