% Options for packages loaded elsewhere
\PassOptionsToPackage{unicode}{hyperref}
\PassOptionsToPackage{hyphens}{url}
%
\documentclass[
  ignorenonframetext,
]{beamer}
\usepackage{pgfpages}
\setbeamertemplate{caption}[numbered]
\setbeamertemplate{caption label separator}{: }
\setbeamercolor{caption name}{fg=normal text.fg}
\beamertemplatenavigationsymbolsempty
% Prevent slide breaks in the middle of a paragraph
\widowpenalties 1 10000
\raggedbottom
\setbeamertemplate{part page}{
  \centering
  \begin{beamercolorbox}[sep=16pt,center]{part title}
    \usebeamerfont{part title}\insertpart\par
  \end{beamercolorbox}
}
\setbeamertemplate{section page}{
  \centering
  \begin{beamercolorbox}[sep=12pt,center]{part title}
    \usebeamerfont{section title}\insertsection\par
  \end{beamercolorbox}
}
\setbeamertemplate{subsection page}{
  \centering
  \begin{beamercolorbox}[sep=8pt,center]{part title}
    \usebeamerfont{subsection title}\insertsubsection\par
  \end{beamercolorbox}
}
\AtBeginPart{
  \frame{\partpage}
}
\AtBeginSection{
  \ifbibliography
  \else
    \frame{\sectionpage}
  \fi
}
\AtBeginSubsection{
  \frame{\subsectionpage}
}
\usepackage{amsmath,amssymb}
\usepackage{lmodern}
\usepackage{iftex}
\ifPDFTeX
  \usepackage[T1]{fontenc}
  \usepackage[utf8]{inputenc}
  \usepackage{textcomp} % provide euro and other symbols
\else % if luatex or xetex
  \usepackage{unicode-math}
  \defaultfontfeatures{Scale=MatchLowercase}
  \defaultfontfeatures[\rmfamily]{Ligatures=TeX,Scale=1}
\fi
% Use upquote if available, for straight quotes in verbatim environments
\IfFileExists{upquote.sty}{\usepackage{upquote}}{}
\IfFileExists{microtype.sty}{% use microtype if available
  \usepackage[]{microtype}
  \UseMicrotypeSet[protrusion]{basicmath} % disable protrusion for tt fonts
}{}
\makeatletter
\@ifundefined{KOMAClassName}{% if non-KOMA class
  \IfFileExists{parskip.sty}{%
    \usepackage{parskip}
  }{% else
    \setlength{\parindent}{0pt}
    \setlength{\parskip}{6pt plus 2pt minus 1pt}}
}{% if KOMA class
  \KOMAoptions{parskip=half}}
\makeatother
\usepackage{xcolor}
\IfFileExists{xurl.sty}{\usepackage{xurl}}{} % add URL line breaks if available
\IfFileExists{bookmark.sty}{\usepackage{bookmark}}{\usepackage{hyperref}}
\hypersetup{
  hidelinks,
  pdfcreator={LaTeX via pandoc}}
\urlstyle{same} % disable monospaced font for URLs
\newif\ifbibliography
\usepackage{color}
\usepackage{fancyvrb}
\newcommand{\VerbBar}{|}
\newcommand{\VERB}{\Verb[commandchars=\\\{\}]}
\DefineVerbatimEnvironment{Highlighting}{Verbatim}{commandchars=\\\{\}}
% Add ',fontsize=\small' for more characters per line
\newenvironment{Shaded}{}{}
\newcommand{\AlertTok}[1]{\textcolor[rgb]{1.00,0.00,0.00}{\textbf{#1}}}
\newcommand{\AnnotationTok}[1]{\textcolor[rgb]{0.38,0.63,0.69}{\textbf{\textit{#1}}}}
\newcommand{\AttributeTok}[1]{\textcolor[rgb]{0.49,0.56,0.16}{#1}}
\newcommand{\BaseNTok}[1]{\textcolor[rgb]{0.25,0.63,0.44}{#1}}
\newcommand{\BuiltInTok}[1]{#1}
\newcommand{\CharTok}[1]{\textcolor[rgb]{0.25,0.44,0.63}{#1}}
\newcommand{\CommentTok}[1]{\textcolor[rgb]{0.38,0.63,0.69}{\textit{#1}}}
\newcommand{\CommentVarTok}[1]{\textcolor[rgb]{0.38,0.63,0.69}{\textbf{\textit{#1}}}}
\newcommand{\ConstantTok}[1]{\textcolor[rgb]{0.53,0.00,0.00}{#1}}
\newcommand{\ControlFlowTok}[1]{\textcolor[rgb]{0.00,0.44,0.13}{\textbf{#1}}}
\newcommand{\DataTypeTok}[1]{\textcolor[rgb]{0.56,0.13,0.00}{#1}}
\newcommand{\DecValTok}[1]{\textcolor[rgb]{0.25,0.63,0.44}{#1}}
\newcommand{\DocumentationTok}[1]{\textcolor[rgb]{0.73,0.13,0.13}{\textit{#1}}}
\newcommand{\ErrorTok}[1]{\textcolor[rgb]{1.00,0.00,0.00}{\textbf{#1}}}
\newcommand{\ExtensionTok}[1]{#1}
\newcommand{\FloatTok}[1]{\textcolor[rgb]{0.25,0.63,0.44}{#1}}
\newcommand{\FunctionTok}[1]{\textcolor[rgb]{0.02,0.16,0.49}{#1}}
\newcommand{\ImportTok}[1]{#1}
\newcommand{\InformationTok}[1]{\textcolor[rgb]{0.38,0.63,0.69}{\textbf{\textit{#1}}}}
\newcommand{\KeywordTok}[1]{\textcolor[rgb]{0.00,0.44,0.13}{\textbf{#1}}}
\newcommand{\NormalTok}[1]{#1}
\newcommand{\OperatorTok}[1]{\textcolor[rgb]{0.40,0.40,0.40}{#1}}
\newcommand{\OtherTok}[1]{\textcolor[rgb]{0.00,0.44,0.13}{#1}}
\newcommand{\PreprocessorTok}[1]{\textcolor[rgb]{0.74,0.48,0.00}{#1}}
\newcommand{\RegionMarkerTok}[1]{#1}
\newcommand{\SpecialCharTok}[1]{\textcolor[rgb]{0.25,0.44,0.63}{#1}}
\newcommand{\SpecialStringTok}[1]{\textcolor[rgb]{0.73,0.40,0.53}{#1}}
\newcommand{\StringTok}[1]{\textcolor[rgb]{0.25,0.44,0.63}{#1}}
\newcommand{\VariableTok}[1]{\textcolor[rgb]{0.10,0.09,0.49}{#1}}
\newcommand{\VerbatimStringTok}[1]{\textcolor[rgb]{0.25,0.44,0.63}{#1}}
\newcommand{\WarningTok}[1]{\textcolor[rgb]{0.38,0.63,0.69}{\textbf{\textit{#1}}}}
\setlength{\emergencystretch}{3em} % prevent overfull lines
\providecommand{\tightlist}{%
  \setlength{\itemsep}{0pt}\setlength{\parskip}{0pt}}
\setcounter{secnumdepth}{-\maxdimen} % remove section numbering
\ifLuaTeX
  \usepackage{selnolig}  % disable illegal ligatures
\fi

\author{}
\date{}

\begin{document}

\begin{frame}[fragile]
\begin{Shaded}
\begin{Highlighting}[]
\FunctionTok{Campus}\KeywordTok{:}\AttributeTok{ Ciudad Universitaria}
\FunctionTok{Facultad}\KeywordTok{:}\AttributeTok{ Ingeniería}
\FunctionTok{Materia }\KeywordTok{:}\AttributeTok{ Inteligencia Artificial}
\FunctionTok{Semestre}\KeywordTok{:}\AttributeTok{ 2022{-}2}
\FunctionTok{Equipo}\KeywordTok{:}\AttributeTok{ }\DecValTok{1}
\FunctionTok{Clave}\KeywordTok{:}\AttributeTok{ }\DecValTok{0406}
\FunctionTok{Participantes}\KeywordTok{:}\AttributeTok{ }
\AttributeTok{    }\KeywordTok{{-}}\AttributeTok{ Barrera Peña Víctor Miguel}
\AttributeTok{    }\KeywordTok{{-}}\AttributeTok{ Espino De Horta Joaquín Gustavo}
\AttributeTok{    }
\FunctionTok{Profesor}\KeywordTok{:}\AttributeTok{ Dr. Ismael Everardo Barcenas Patiño}
\FunctionTok{Título }\KeywordTok{:}\AttributeTok{ Proyecto 2}
\FunctionTok{Subtítulo }\KeywordTok{:}\AttributeTok{ Fórmula de lógica proposicional valida o no}
\end{Highlighting}
\end{Shaded}
\end{frame}

\begin{frame}{Definición del problema}
\protect\hypertarget{definiciuxf3n-del-problema}{}
\end{frame}

\begin{frame}[fragile]{Capítulo 0 Estructura del repositorio}
\protect\hypertarget{capuxedtulo-0-estructura-del-repositorio}{}
El repositorio documenta todo el proceso hecho para realizar el programa
, incluyendo los pdf´s

\begin{block}{Bibliotecas usadas}
\protect\hypertarget{bibliotecas-usadas}{}
\begin{Shaded}
\begin{Highlighting}[]
\FunctionTok{\textbackslash{}usespackage}\NormalTok{\{prof\}}
\FunctionTok{\textbackslash{}usespackage}\NormalTok{\{bussproofs\}}
\end{Highlighting}
\end{Shaded}

Para poder ejecutar el programa es necesario tener instalado un
interprete de \texttt{prolog} para poder ejecutar el programa.
\end{block}
\end{frame}

\begin{frame}{Capítulo 1 Introducción}
\protect\hypertarget{capuxedtulo-1-introducciuxf3n}{}
\end{frame}

\begin{frame}{Capítulo 2 Desarrollo}
\protect\hypertarget{capuxedtulo-2-desarrollo}{}
\begin{block}{Idea de desarrollo del programa}
\protect\hypertarget{idea-de-desarrollo-del-programa}{}
\end{block}

\begin{block}{Casos de prueba}
\protect\hypertarget{casos-de-prueba}{}
\begin{block}{Triviales (1 caso)}
\protect\hypertarget{triviales-1-caso}{}
\end{block}

\begin{block}{Fáciles (3 casos)}
\protect\hypertarget{fuxe1ciles-3-casos}{}
\end{block}

\begin{block}{Media (3 casos)}
\protect\hypertarget{media-3-casos}{}
\end{block}

\begin{block}{Difíciles ( 3 casos )}
\protect\hypertarget{difuxedciles-3-casos}{}
\end{block}

\begin{block}{Sin solución (1 caso)}
\protect\hypertarget{sin-soluciuxf3n-1-caso}{}
\end{block}

\begin{block}{Código}
\protect\hypertarget{cuxf3digo}{}
\end{block}
\end{block}

\begin{block}{Explicación código}
\protect\hypertarget{explicaciuxf3n-cuxf3digo}{}
\end{block}
\end{frame}

\begin{frame}{Capítulo 3 Conclusión}
\protect\hypertarget{capuxedtulo-3-conclusiuxf3n}{}
\begin{block}{Barrera Peña Víctor Miguel}
\protect\hypertarget{barrera-peuxf1a-vuxedctor-miguel}{}
\end{block}

\begin{block}{Espino de Horta Joaquín Gustavo}
\protect\hypertarget{espino-de-horta-joaquuxedn-gustavo}{}
\end{block}
\end{frame}

\end{document}
